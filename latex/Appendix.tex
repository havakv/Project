\chapter{Methods for validation}
\label{chap:Metods for validation}
\todo{Find better name}
\section{EPE}
\label{sec:EPE}
The engine in prediction problems is to minimize the expected prediction error $\mathrm{EPE}$ of some function $\hat f$, under some loss function $L$
\\\colorbox{yellow}{Find good notation $\hat f$ when comparing with sections.}
\begin{align}
  \mathrm{EPE}(\hat f) = \E_{X Y}[L(Y, \hat f(X))] = \E_X [\E_Y (L(Y, \hat f(X))|X)].
\end{align}
Thus 
\begin{align}
  f = \argmin_{\hat f} \mathrm{EPE}(\hat f) = \argmin_{\hat f} \E_Y [L(Y, \hat f(X)) | X ].
\end{align}
Under squared error loss, $L(Y, \hat f(X)) = (Y- \hat f(X)^2)$, the solution is 
$ f(x) = \E[Y | X=x]$. And under $0/1$ loss $ f(x)$ is the \textit{Bayes classifier} 
\begin{align}
  f(x) = \E[Y | X=x].
\end{align}

\subsection{Bias-variance tradeoff}
\label{sub:Bias-variance tradeoff}
In the regression framework the relation,
\begin{align}
  Y = f(X) + \varepsilon,  
\end{align}
is assumed, where $\varepsilon$ has zero mean and variance $\sigma^2$.
Under this assumption $\mathrm{EPE}$ can be decomposed into
\begin{align}
  \mathrm{EPE}(\hat f) &=  \E_Y[Y^2] + \E_X[X^2] - \E_{X Y}[2 y \hat f(X)] \notag \\
                       &= \Var[Y] + \E_Y[Y]^2 + \Var[X] + \E_X[X]^2 - 2 f\: \E_X[\hat f(X)] \notag \\
                       &= \Var[Y] + \Var[\hat f(X)] + (f - \E_X[\hat f(X)])^2 \notag \\
                       &= \sigma^2 + \Var[\hat f(X)] + \mathrm{Bias}[\hat f(X)]^2.
\end{align}
So to minimize $\mathrm{EPE}(\hat f)$ it is necessary to minimize the variance and bias of $\hat f$. This equation is often referred to as the bias-variance tradeoff as usually when one is decreased the other increase. 

As discussed in \ref{sub:Classification trees}, there is no equally satisfactory decomposition of $\mathrm{EPE}$ under $0/1$-loss, but it is still common to talk about the bias-variance tradeoff in classification as well.



\section{Cross-validation}
\label{sec:Cross-validation}

\section{Bootstrapping}
\label{sec:Bootstrapping}
Bootstrapping refers to methods based on random sampling with replacement. It is used to approximate the distribution of data, based on the data itself. One can thus do inference on data, like for instance a confidence interval for a parameter estimate. It can also be used to improve predictions, like in the Bagging algorithm in Section~\ref{sec:Bagging}. In the context of this project, bootstrapping refers to sample $N$ data points from the dataset $\left\{ \mathbf{x}_i, y_i \right\}_{i = 1}^N$, with replacement, and is only used for improving predictions. 
For an introduction to bootstrapping, see \cite{efron1994bootstrap}.


\section{Variance for random forest regression}
\label{sec:Variance for random forest regression}
Let $T_i(\mathbf{x})$ denote a trained tree. The random forest prediction for $\mathbf{x}$ is 
\begin{align}
  \hat f(\mathbf{x}) = \frac{1}{B} \sum_{i = 1}^{B} T_i(\mathbf{x}).
\end{align}
The trees are created by drawing from the same distributions, so they should be identically distributed. \todo{Is this correct?}
\begin{align}
  \Cov[T_i(\mathbf{x}), T_j(\mathbf{x}))] = \rho \sigma^2, \quad \quad i \neq j,
\end{align}
where $\sigma^2$ is the variance of a tree and $\rho$ is the correlation between trees.  
The variance of $\hat f(\mathbf{x})$ is thus,
\begin{align}
\Var\left[ \frac{1}{B} \sum_{i = 1}^{B} T_i(\mathbf{x}) \right] 
&= \frac{1}{B^2} \sum_{i =1}^B \sum_{j= 1}^B \Cov[T_i(\mathbf{x}), T_j(\mathbf{x})] \notag \\
&= \frac{1}{B^2} \sum_{i =1}^B \left(\sum_{j \neq i} \Cov[T_i(\mathbf{x}), T_j(\mathbf{x})] + \Var[T_i(\mathbf{x})]  \right)\notag \\
&= \frac{1}{B^2} \sum_{i =1}^B \left( (B-1) \sigma^2 \rho + \sigma^2  \right)\notag \\
&= \rho \sigma^2 + \sigma^2 \frac{1-\rho}{B}.
\end{align}
So it decrease by increasing the number of bootstrap samples $B$ and by decorrelating the trees.
